\documentclass[journal,12pt,twocolumn]{IEEEtran}
%
\usepackage{setspace}
\usepackage{gensymb}
%\doublespacing
\singlespacing

%\usepackage{graphicx}
%\usepackage{amssymb}
%\usepackage{relsize}
\usepackage[cmex10]{amsmath}
%\usepackage{amsthm}
%\interdisplaylinepenalty=2500
%\savesymbol{iint}
%\usepackage{txfonts}
%\restoresymbol{TXF}{iint}
%\usepackage{wasysym}
\usepackage{amsthm}
%\usepackage{iithtlc}
\usepackage{mathrsfs}
\usepackage{txfonts}
\usepackage{stfloats}
\usepackage{bm}
\usepackage{cite}
\usepackage{cases}
\usepackage{subfig}
%\usepackage{xtab}
\usepackage{longtable}
\usepackage{multirow}
%\usepackage{algorithm}
%\usepackage{algpseudocode}
\usepackage{enumitem}
\usepackage{mathtools}
\usepackage{steinmetz}
\usepackage{tikz}
\usepackage{circuitikz}
\usepackage{verbatim}
\usepackage{tfrupee}
\usepackage[breaklinks=true]{hyperref}
%\usepackage{stmaryrd}
\usepackage{tkz-euclide} % loads  TikZ and tkz-base
%\usetkzobj{all}
\usetikzlibrary{calc,math}
\usepackage{listings}
    \usepackage{color}                                            %%
    \usepackage{array}                                            %%
    \usepackage{longtable}                                        %%
    \usepackage{calc}                                             %%
    \usepackage{multirow}                                         %%
    \usepackage{hhline}                                           %%
    \usepackage{ifthen}                                           %%
  %optionally (for landscape tables embedded in another document): %%
    \usepackage{lscape}     
\usepackage{multicol}
\usepackage{chngcntr}
%\usepackage{enumerate}

%\usepackage{wasysym}
%\newcounter{MYtempeqncnt}
\DeclareMathOperator*{\Res}{Res}
%\renewcommand{\baselinestretch}{2}
\renewcommand\thesection{\arabic{section}}
\renewcommand\thesubsection{\thesection.\arabic{subsection}}
\renewcommand\thesubsubsection{\thesubsection.\arabic{subsubsection}}

\renewcommand\thesectiondis{\arabic{section}}
\renewcommand\thesubsectiondis{\thesectiondis.\arabic{subsection}}
\renewcommand\thesubsubsectiondis{\thesubsectiondis.\arabic{subsubsection}}

% correct bad hyphenation here
\hyphenation{op-tical net-works semi-conduc-tor}
\def\inputGnumericTable{}                                 %%

\lstset{
%language=C,
frame=single, 
breaklines=true,
columns=fullflexible
}
%\lstset{
%language=tex,
%frame=single, 
%breaklines=true
%}

\begin{document}
%


\newtheorem{theorem}{Theorem}[section]
\newtheorem{problem}{Problem}
\newtheorem{proposition}{Proposition}[section]
\newtheorem{lemma}{Lemma}[section]
\newtheorem{corollary}[theorem]{Corollary}
\newtheorem{example}{Example}[section]
\newtheorem{definition}[problem]{Definition}
%\newtheorem{thm}{Theorem}[section] 
%\newtheorem{defn}[thm]{Definition}
%\newtheorem{algorithm}{Algorithm}[section]
%\newtheorem{cor}{Corollary}
\newcommand{\BEQA}{\begin{eqnarray}}
\newcommand{\EEQA}{\end{eqnarray}}
\newcommand{\define}{\stackrel{\triangle}{=}}
\bibliographystyle{IEEEtran}
%\bibliographystyle{ieeetr}
\providecommand{\mbf}{\mathbf}
\providecommand{\pr}[1]{\ensuremath{\Pr\left(#1\right)}}
\providecommand{\qfunc}[1]{\ensuremath{Q\left(#1\right)}}
\providecommand{\sbrak}[1]{\ensuremath{{}\left[#1\right]}}
\providecommand{\lsbrak}[1]{\ensuremath{{}\left[#1\right.}}
\providecommand{\rsbrak}[1]{\ensuremath{{}\left.#1\right]}}
\providecommand{\brak}[1]{\ensuremath{\left(#1\right)}}
\providecommand{\lbrak}[1]{\ensuremath{\left(#1\right.}}
\providecommand{\rbrak}[1]{\ensuremath{\left.#1\right)}}
\providecommand{\cbrak}[1]{\ensuremath{\left\{#1\right\}}}
\providecommand{\lcbrak}[1]{\ensuremath{\left\{#1\right.}}
\providecommand{\rcbrak}[1]{\ensuremath{\left.#1\right\}}}
\theoremstyle{remark}
\newtheorem{rem}{Remark}
\newcommand{\sgn}{\mathop{\mathrm{sgn}}}
\providecommand{\abs}[1]{$\left\vert#1\right\vert$}
\providecommand{\res}[1]{\Res\displaylimits_{#1}} 
\providecommand{\norm}[1]{$\left\lVert#1\right\rVert$}
%\providecommand{\norm}[1]{\lVert#1\rVert}
\providecommand{\mtx}[1]{\mathbf{#1}}
\providecommand{\mean}[1]{E$\left[ #1 \right$]}
\providecommand{\fourier}{\overset{\mathcal{F}}{ \rightleftharpoons}}
%\providecommand{\hilbert}{\overset{\mathcal{H}}{ \rightleftharpoons}}
\providecommand{\system}{\overset{\mathcal{H}}{ \longleftrightarrow}}
	%\newcommand{\solution}[2]{\textbf{Solution:}{#1}}
\newcommand{\solution}{\noindent \textbf{Solution: }}
\newcommand{\cosec}{\,\text{cosec}\,}
\providecommand{\dec}[2]{\ensuremath{\overset{#1}{\underset{#2}{\gtrless}}}}
\newcommand{\myvec}[1]{\ensuremath{\begin{pmatrix}#1\end{pmatrix}}}
\newcommand{\mydet}[1]{\ensuremath{\begin{vmatrix}#1\end{vmatrix}}}
%\numberwithin{equation}{section}
\numberwithin{equation}{subsection}
%\numberwithin{problem}{section}
%\numberwithin{definition}{section}
\makeatletter
\@addtoreset{figure}{problem}
\makeatother
\let\StandardTheFigure\thefigure
\let\vec\mathbf
%\renewcommand{\thefigure}{\theproblem.\arabic{figure}}
\renewcommand{\thefigure}{\theproblem}
%\setlist[enumerate,1]{before=\renewcommand\theequation{\theenumi.\arabic{equation}}
%\counterwithin{equation}{enumi}
%\renewcommand{\theequation}{\arabic{subsection}.\arabic{equation}}
\def\putbox#1#2#3{\makebox[0in][l]{\makebox[#1][l]{}\raisebox{\baselineskip}[0in][0in]{\raisebox{#2}[0in][0in]{#3}}}}
     \def\rightbox#1{\makebox[0in][r]{#1}}
     \def\centbox#1{\makebox[0in]{#1}}
     \def\topbox#1{\raisebox{-\baselineskip}[0in][0in]{#1}}
     \def\midbox#1{\raisebox{-0.5\baselineskip}[0in][0in]{#1}}
\vspace{3cm}
\title{Matrix theory Assignment 1}
\author{SREEKANTH SANKALA}

\maketitle
\newpage
\bigskip
\renewcommand{\thefigure}{\theenumi}
\renewcommand{\thetable}{\theenumi}

\begin{abstract}
This document contains the procedure to get a equation of a line that is equidistant from the two parallel lines.
\end{abstract}
Download the python code from the below link. Go through the README file in the reposotory.
%
\begin{lstlisting}
https://github.com/Sreeakanth/EE5609
\end{lstlisting}
%
\begin{comment}
and latex-tikz codes from 
%
\begin{lstlisting}
https://github.com/saipranavkr/EE5609
\end{lstlisting}
%
\end{comment}
\section{Problem}

Find the equation of the line which is equidistant from two parallel lines. 
\begin{equation}
    (9\;\;7)\textbf{x} = 7 
\end{equation}
\begin{equation}
    (3\;\;2)\textbf{x} = -6 
\end{equation}

I think there is a typo in the problem , I think they are non parallel lines, However the procedure is similar.

Assume that an arbitrary point $(x'\;\;y')$ is on the resultant line. Then the Idea of computing the line that is equidistant from two other lines is compute the distant between an arbitrary point $(x'\;\;y')$ to both lines and equate then with opposite signs.

the distance from arbitrary point $(x'\;\;y')$ to line L1 is same as L2 with opposite sign. Let's $ax+by+c=0$ and $dx+ey+f=0$ be the two given lines then
\begin{equation}
    \frac{ax'+by'+c}{\sqrt{{a}^2 +{b}^2}} = - \frac{dx'+ey'+f}{\sqrt{{d}^2 +{e}^2}}
\end{equation}
When we reformulate the above equation to the general form of linear equation we will get.

\begin{equation}
    (\frac{a}{{\sqrt{{a}^2 +{b}^2}}} + \frac{d}{{\sqrt{{d}^2 +{e}^2}}})(x') + (\frac{b}{{\sqrt{{a}^2 +{b}^2}}} + \frac{e}{{\sqrt{{d}^2 +{e}^2}}})(y') + (\frac{c}{{\sqrt{{a}^2 +{b}^2}}} + \frac{f}{{\sqrt{{d}^2 +{e}^2}}}) = 0
\end{equation}

So the coefficients of the results line gx'+hy'+i=0 are
\begin{equation}
    g=(\frac{a}{{\sqrt{{a}^2 +{b}^2}}} + \frac{d}{{\sqrt{{d}^2 +{e}^2}}})
\end{equation}
 
\begin{equation}
    h=(\frac{b}{{\sqrt{{a}^2 +{b}^2}}} + \frac{e}{{\sqrt{{d}^2 +{e}^2}}})
\end{equation}

\begin{equation}
    i=(\frac{c}{{\sqrt{{a}^2 +{b}^2}}} + \frac{f}{{\sqrt{{d}^2 +{e}^2}}})
\end{equation}

When we substitute the given data in the above equations we will get the answer. given data is
$a=9$,$b=7$,$c=-7$ and $d=3$,$e=2$,$f=6$
so the answer is 
\begin{equation}
    g= 1.62
\end{equation}
\begin{equation}
    h= 1.168
\end{equation}
\begin{equation}
    i= 1.05
\end{equation}

I also took parallel lines and then compute the line which is equidistant from the two parallel lines

\begin{equation}
    (2\;\;4)\textbf{x} = 8
\end{equation}
\begin{equation}
    (2\;\;4)\textbf{x} = 16
\end{equation}

Now the coefficient of the line gx+hy+i=0 as mentioned by the above procedure is.

\begin{equation}
    g= 0.89
\end{equation}
\begin{equation}
    h= 1.78
\end{equation}
\begin{equation}
    i= -5.36
\end{equation}
 
You can see the slopes of all the three lines in the above examples are similar (-0.5).

\end{document}
